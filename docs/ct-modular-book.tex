% Options for packages loaded elsewhere
\PassOptionsToPackage{unicode}{hyperref}
\PassOptionsToPackage{hyphens}{url}
%
\documentclass[
]{book}
\usepackage{amsmath,amssymb}
\usepackage{lmodern}
\usepackage{iftex}
\ifPDFTeX
  \usepackage[T1]{fontenc}
  \usepackage[utf8]{inputenc}
  \usepackage{textcomp} % provide euro and other symbols
\else % if luatex or xetex
  \usepackage{unicode-math}
  \defaultfontfeatures{Scale=MatchLowercase}
  \defaultfontfeatures[\rmfamily]{Ligatures=TeX,Scale=1}
\fi
% Use upquote if available, for straight quotes in verbatim environments
\IfFileExists{upquote.sty}{\usepackage{upquote}}{}
\IfFileExists{microtype.sty}{% use microtype if available
  \usepackage[]{microtype}
  \UseMicrotypeSet[protrusion]{basicmath} % disable protrusion for tt fonts
}{}
\makeatletter
\@ifundefined{KOMAClassName}{% if non-KOMA class
  \IfFileExists{parskip.sty}{%
    \usepackage{parskip}
  }{% else
    \setlength{\parindent}{0pt}
    \setlength{\parskip}{6pt plus 2pt minus 1pt}}
}{% if KOMA class
  \KOMAoptions{parskip=half}}
\makeatother
\usepackage{xcolor}
\usepackage{longtable,booktabs,array}
\usepackage{calc} % for calculating minipage widths
% Correct order of tables after \paragraph or \subparagraph
\usepackage{etoolbox}
\makeatletter
\patchcmd\longtable{\par}{\if@noskipsec\mbox{}\fi\par}{}{}
\makeatother
% Allow footnotes in longtable head/foot
\IfFileExists{footnotehyper.sty}{\usepackage{footnotehyper}}{\usepackage{footnote}}
\makesavenoteenv{longtable}
\usepackage{graphicx}
\makeatletter
\def\maxwidth{\ifdim\Gin@nat@width>\linewidth\linewidth\else\Gin@nat@width\fi}
\def\maxheight{\ifdim\Gin@nat@height>\textheight\textheight\else\Gin@nat@height\fi}
\makeatother
% Scale images if necessary, so that they will not overflow the page
% margins by default, and it is still possible to overwrite the defaults
% using explicit options in \includegraphics[width, height, ...]{}
\setkeys{Gin}{width=\maxwidth,height=\maxheight,keepaspectratio}
% Set default figure placement to htbp
\makeatletter
\def\fps@figure{htbp}
\makeatother
\setlength{\emergencystretch}{3em} % prevent overfull lines
\providecommand{\tightlist}{%
  \setlength{\itemsep}{0pt}\setlength{\parskip}{0pt}}
\setcounter{secnumdepth}{5}
\usepackage{booktabs}
\ifLuaTeX
  \usepackage{selnolig}  % disable illegal ligatures
\fi
\usepackage[]{natbib}
\bibliographystyle{apalike}
\IfFileExists{bookmark.sty}{\usepackage{bookmark}}{\usepackage{hyperref}}
\IfFileExists{xurl.sty}{\usepackage{xurl}}{} % add URL line breaks if available
\urlstyle{same} % disable monospaced font for URLs
\hypersetup{
  pdftitle={Computational Thinking through Modular Sounds Synthesis},
  pdfauthor={Andrew M. Olney},
  hidelinks,
  pdfcreator={LaTeX via pandoc}}

\title{Computational Thinking through Modular Sounds Synthesis}
\author{Andrew M. Olney}
\date{2022-08-21}

\begin{document}
\maketitle

{
\setcounter{tocdepth}{1}
\tableofcontents
}
\hypertarget{welcome}{%
\chapter*{Welcome}\label{welcome}}
\addcontentsline{toc}{chapter}{Welcome}

This is the official website for ``Computational Thinking through Modular Sound Synthesis''. This book will teach you computational thinking through modular sound synthesis (hereafter \emph{modular}). You'll learn how to trigger sounds, create sounds, and modify sounds to solve specific sound design problems and create compositions. Along the way, you'll learn computational thinking practices that transcend modular and can be applied to a variety of problem-solving domains, but which are particularly relevant to information processing domains like computing.

If you're wondering whether this is a book about computational thinking, or a book about modular, the answer is both: on the surface, most content is about modular, but computational thinking is a style of thinking reflected in the presentation of the material and gives it additional coherence. As you work through the book, you'll become more proficient in computational thinking practices like decomposition, algorithmic design, evaluation of solutions, pattern recognition, and abstraction.

This book is \emph{interactive}, which is why it is an e-book rather than a paper book. Throughout you will encounter examples, simulations, and exercises that run in your browser to demonstrate and reinforce key concepts. Don't skip the interactive activities!

\includegraphics{images/by-nc-nd.png}

This website is free to use and is licensed under the \href{https://creativecommons.org/licenses/by-nc-nd/4.0/}{Creative Commons Attribution-NonCommercial-NoDerivs 4.0 License}.

\hypertarget{why-this-book}{%
\chapter{Why this book?}\label{why-this-book}}

\hypertarget{part-sound}{%
\part{Sound}\label{part-sound}}

\hypertarget{physics-and-perception}{%
\chapter{Physics and Perception}\label{physics-and-perception}}

\hypertarget{harmonic-sounds}{%
\chapter{Harmonic Sounds}\label{harmonic-sounds}}

\hypertarget{inharmonic-sounds}{%
\chapter{Inharmonic Sounds}\label{inharmonic-sounds}}

\hypertarget{part-fundamental-modules}{%
\part{Fundamental Modules}\label{part-fundamental-modules}}

\hypertarget{basic-concepts}{%
\chapter{Basic Concepts}\label{basic-concepts}}

\hypertarget{trigger}{%
\chapter{Trigger}\label{trigger}}

\hypertarget{create}{%
\chapter{Create}\label{create}}

\hypertarget{modify}{%
\chapter{Modify}\label{modify}}

\hypertarget{part-sound-design-1}{%
\part{Sound Design 1}\label{part-sound-design-1}}

\hypertarget{kick-cymbal}{%
\chapter{Kick \& Cymbal}\label{kick-cymbal}}

\hypertarget{lead-bass}{%
\chapter{Lead \& Bass}\label{lead-bass}}

\hypertarget{part-complex-modules}{%
\part{Complex Modules}\label{part-complex-modules}}

\hypertarget{trigger-1}{%
\chapter{Trigger}\label{trigger-1}}

\hypertarget{create-1}{%
\chapter{Create}\label{create-1}}

\hypertarget{modify-1}{%
\chapter{Modify}\label{modify-1}}

\hypertarget{part-sound-design-2}{%
\part{Sound Design 2}\label{part-sound-design-2}}

\hypertarget{minimoog-303}{%
\chapter{Minimoog \& 303}\label{minimoog-303}}

  \bibliography{book.bib,packages.bib}

\end{document}
